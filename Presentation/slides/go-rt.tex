\documentclass{beamer}
\usepackage[UTF8,noindent,adobefonts]{ctex}
\usepackage{graphicx}
\usepackage{amsmath}

\usefonttheme[onlymath]{serif}
\usetheme{Boadilla}
%\usetheme{Madrid}

\title{Go语言的运行时系统}
\author{阴钰 <yxonic@gmail.com>}
\date{}

\begin{document}

% Customized title page.
\begin{frame}
  \vfill
  \quad\textbf{\Huge Go语言的运行时系统}
  \vfill
  \quad 阴\quad 钰\quad \texttt{PB13011038 \scriptsize yxonic@gmail.com}
\end{frame}

\section{Go基础}

\begin{frame}{\bf What is Go}

  Go,又称golang,是Google开发的一种\textbf{静态强类型}、\textbf{编译
    型},\textbf{并发型},并具有\textbf{垃圾回收}功能的\textbf{系统
    级}编程语言。

\end{frame}

\begin{frame}{\bf Go in real world}

\texttt{https://www.zhihu.com/question/21409296/answer/18184584}
  
\end{frame}

\begin{frame}{\bf Go basics}
  \begin{itemize}
  \item 变量、函数声明
  \item 类型系统
  \item goroutines \& channels
  \end{itemize}
\end{frame}

\section{Go Runtime}

\begin{frame}{\bf Go代码概览}

\begin{itemize}
\pause
\item 主要代码位于\texttt{src/runtime}下。
\pause
\item C、ASM与Go的混合代码。
\pause
\item 汇编语言的约定、连接器的约定。
\pause
\item 手段:编译到汇编代码,结合文档、注释阅读用到的函数。
\end{itemize}

\end{frame}

\begin{frame}{\bf 调度器}

\begin{itemize}
\pause
\item Go语言内置一个抢占式的调度器,用来调度goroutine。
\pause
\item 启动时配置好调度器,运行第一个goroutine,由它负责开启其他goroutine(包
括main)。
\pause
\item 三个组件:G, M, P。
\pause
\item 机制
\end{itemize}

\end{frame}

\begin{frame}{\bf Channel}

\begin{itemize}
\pause
\item 创建
\pause
\item 收发
\pause
\item Select
\end{itemize}

\end{frame}

\end{document}
