% -*- coding: utf-8 -*-
% -- by Yin Yu --
\documentclass[UTF8,b5paper,fontset=adobe]{ctexart}
\usepackage{hyperref}
\hypersetup{
  unicode=true,
  pdfstartview={FitH},
  pdftitle={Notes on Go Runtime},
  pdfauthor={阴钰},
  colorlinks=true,
  linkcolor=blue,
  citecolor=cyan
}
\pagestyle{plain}

\title{\textbf{Notes on Go Runtime}}
\author{\texttt{计算机科学与技术学院\quad PB13011038} \\
  \textit{阴钰} \\
  \texttt{\href{mailto:yxonic@gmail.com}
    {yxonic@gmail.com}}}
\date{}

\bibliographystyle{plain}
\usepackage{fullpage}
\begin{document}

\maketitle

\renewcommand\contentsname{目 录}
\tableofcontents

\section{Go的特色}

Go,又称golang,是Google开发的一种\textbf{静态强类型},\textbf{编译
  型},\textbf{并发型},并具有\textbf{垃圾回收}功能的\textbf{系统级}编
程语言\cite{pike2009}。

Go区别于其他语言的主要特色,集中体现在Go的运行时系统上,尤其是运行时的
高效的调度、栈空间管理、同步通信。本文主要就这些内容,对Go的源代码进行
分块解析。

\section{准备工作}

采用的Go源码为1.5.3版。对于本文讨论的内容,具观察,从1.4到1.5,实现思路
都变化不大,大部分算法、甚至函数名,都几乎没有变化;1.5版本后运行时也全
部采用Go和汇编,去掉了C代码,使得代码更为一致,因此本文主要对1.5的代码
进行分析。

要了解运行时各函数的作用、何时被调用,我们还需要对编译好的Go程序的结构
有所了解。使用如下命令将Go程序编译到Plan9的汇编代码(这是一种通用汇编
码,可以被Go的汇编器转为常见架构的机器码):

\begin{verbatim}
    $ go tool 6g -S hello.go
\end{verbatim}

得到的汇编码,以及标准库中的汇编码,都遵循相同的格式。这种汇编代码中,
有几个符号需要注意:SB (static base)用来表示静态数据区的基址,FP
(frame pointer)用来表示函数参数基址,SP (stack pointer)表示局部数据的基
址。用foo(SB)表示一个叫foo的全局名字,foo<>(SB)表示一个static的全局名
字,0(FP), 8(FP)分别表示第一个、第二个参数(64位系统上),x-8(SP)表示位
于-8位置的叫x的局部名字。

函数、全局变量分别为下面的结构:

\begin{verbatim}
    TEXT runtime·profileloop(SB),NOSPLIT,$8
            MOVQ    $runtime·profileloop1(SB), CX
            MOVQ    CX, 0(SP)
            CALL    runtime·externalthreadhandler(SB)
            RET
        
    DATA symbol+offset(SB)/width, value
\end{verbatim}

另外,为了更方便地了解程序流程,我也采用了GDB调试的方法。使用:
\begin{verbatim}
    $ go build hello.go
\end{verbatim}
得到的程序即带有调试信息,可以直接进行GDB。

Go的runtime代码全部集中于src/runtime下。下文提到的文件如不加说明都是这
个目录下的文件。

本文将只关心linux+amd64环境。其他环境道理是相同的。

\section{Go程序的启动流程}

程序员编写的Go程序的入口为 \verb+main.main+,但是在执行这个函数前,Go运
行时有一些工作要做。在gdb中,把断点设为入口点地址(通过info files获取),
可以知道,程序开始
于 \verb+rt0_linux_amd64.s+中的 \verb+_rt0_amd64_linux+函数:

\begin{verbatim}
    TEXT _rt0_amd64_linux(SB),NOSPLIT,$-8
        LEAQ    8(SP), SI // argv
        MOVQ    0(SP), DI // argc
        MOVQ    $main(SB), AX
        JMPAX             // call main in this file

    TEXT main(SB),NOSPLIT,$-8
        MOVQ    $runtime·rt0_go(SB), AX
        JMPAX             // call runtime.rt0_go
\end{verbatim}

之后调用了 \verb+runtime.rt0_go+,它在 \verb+asm_amd64.s+中。针对CPU做一
些检测后,分别进行了如下工作:为每个CPU初始化两个寄存器变量g0和m0,它们
的作用之后在调度器部分会提到,现在只需要知道g0和m0为寄存器变量,不同处
理器上这些变量并不相同即可;然后处理参数、初始化系统、初始化调度器,都
是对一些全局参数赋值、为一些全局数据结构初始化等操作;然后
以 \verb+runtime.mainPC+为参数,调用 \verb+runtime.newproc+,这个函数的机
制也在调度器部分再考虑,此处只要知道它将 \verb+runtime.mainPC+函数加入调
度队列;然后紧接着的 \verb+runtime.mstart+开启调度
器, \verb+runtime.mainPC+会被调度执行,它将负责调用 \verb+runtime.main+。

\begin{verbatim}
    TEXT runtime·rt0_go(SB),NOSPLIT,$0
        // ...

        // set the per-goroutine and per-mach "registers"
        get_tls(BX)
        LEAQ    runtime·g0(SB), CX
        MOVQ    CX, g(BX)
        LEAQ    runtime·m0(SB), AX

        // save m->g0 = g0
        MOVQ    CX, m_g0(AX)
        // save m0 to g0->m
        MOVQ    AX, g_m(CX)

        CLD     // convention is D is always left cleared
        CALL    runtime·check(SB)

        MOVL    16(SP), AX  // copy argc
        MOVL    AX, 0(SP)
        MOVQ    24(SP), AX  // copy argv
        MOVQ    AX, 8(SP)
        CALL    runtime·args(SB)
        CALL    runtime·osinit(SB)
        CALL    runtime·schedinit(SB)

        // create a new goroutine to start program
        MOVQ    $runtime·mainPC(SB), AX // entry
        PUSHQ   AX
        PUSHQ   $0  // arg size
        CALL    runtime·newproc(SB)
        POPQ    AX
        POPQ    AX

        // start this M
        CALL    runtime·mstart(SB)

        MOVL    $0xf1, 0xf1  // crash
        RET
\end{verbatim}

\section{Go的调度器}

编译一个含有goroutine的Go程序,可以看到go语句被简单地转换为一
个 \verb+newproc+函数。这和之前初始化第一个goroutine时是同一个函数。它负
责将一个函数加入调度队列。在了解它的详细行为之前,我们先整体了解一
下Go的调度器。

Go调度器主要在proc.go中实现,有关机器的细节分散在 \verb+asm_amd64.s+,
 \verb+sys_linux_amd64.s+之中。

Go调度器最重要的三个数据结构分别为G、M和P。

G对应于goroutine,是抽象的任务。M对应系统线程(在linux下即进程),它负
责执行G。G只保存任务本身的信息,执行时必须绑定到M上。而底层的处理器对
应P,M执行时绑定于P之上。这个设计的细节后面还会提到。

调度器的调度算法基于任务窃取。于是,调度器本身的工作比较容易,即找一个
等待中的任务,执行它。整个工作由proc.go中的 \verb+schedule+实现。这个函
数选择一个任务g(通过 \verb+runtime.findrunnable+,这个函数如果在当前队
列找不到任务会从别的队列窃取,有关调度队列的细节在后面有讲述),然后调
用 \verb+execute+,其中调用 \verb+runtime.gogo+实际执行,这个汇编函数会直
接修改栈为g的栈(在这之前将返回地址设为 \verb+goexit+,从而在G结束时做一
些清理工作),从而(继续)执行g。于是,我们开始时需要建立一个定时调
用 \verb+schedule+的守护进程。当然也可以主动调用 \verb+schedule+释放控
制。

这一切都开始于 \verb+runtime.mstart+函数。这个函数在做了一些基本的初始化
之后,调用了 \verb+mstart1+:

\begin{verbatim}
    func mstart1() {
        _g_ := getg()

        if _g_ != _g_.m.g0 {
            throw("bad runtime·mstart")
        }

        // Record top of stack for use by mcall.
        gosave(&_g_.m.g0.sched)
        _g_.m.g0.sched.pc = ^uintptr(0)
        asminit()
        minit()

        // Install signal handlers.
        if _g_.m == &m0 {
            if iscgo && !cgoHasExtraM {
                cgoHasExtraM = true
                newextram()
            }
            initsig()
        }

        // Call startup callbacks.
        if fn := _g_.m.mstartfn; fn != nil {
            fn()
        }

        if _g_.m.helpgc != 0 {
            _g_.m.helpgc = 0
            stopm()
        } else if _g_.m != &m0 {
            acquirep(_g_.m.nextp.ptr())
            _g_.m.nextp = 0
        }
        schedule()
    }
\end{verbatim}

注意到,如果这个M没有和某个P绑定,就会调用 \verb+acquirep+来绑定。在准
备工作做好后,便执行了核心的 \verb+schedule+函数,于是进入调度循环。

下面再详细讨论调度队列的结构。

G的创建(通过 \verb+newproc+),默认会复用之前的G对象,这个空闲的对象来
自P的cache(这是为什么有了M仍然需要P结构的一个理由:M对应的系统进程不一
定在同一个CPU上执行,于是寄存器、cache信息不宜保存于M中)。当然,如果没
有之前的cache,就新建一个G(通过 \verb+malg+分派空间)。特别注明,新G的
栈空间是非常小的(2KB),只有需要时才会扩大,这允许一个程序运行相当大量
的G。

新建好的G被放置到当前P的待运行队列中( \verb+runqput+)。而P的待运行队列
是分级的,分别是下一个执行的任务、P的本地队列、全局队列。前两个由于限定
于本CPU,是不需要加锁访问的。如果P的本地队列满,那么说明P比较忙,这时执
行 \verb+runqputslow+,专门负责把一半任务分到全局队列,没有任务执行的P这
时就可以去执行这些任务了。因此,这一切结束前,应当唤醒空闲的P。

唤醒P调用了 \verb+wakep+函数。它最终在某个P上调用了 \verb+startm+,基本逻
辑为找到一个M,然后执行它。没有空闲M则新建一个(通过 \verb+newm+,正是这
里建立了系统进程,在linux下,此处调用了 \verb+clone+函数)。从此,正常情
况下,这个新的M便会一直跑在这个空闲的P上,进行完整的调度逻辑。然而没有
任务、阻塞过久等情况时,M的P会被夺走,于是这个M便空闲出来。它之后可以被
再次 \verb+mstart+。

对于机制有所了解后,再来看具体的代码就很清楚了。下面是之前提到的重要函
数的有所删减和注释的源码:

\begin{verbatim}
    func schedule() {
        _g_ := getg()

        // ...

        // if there is a locked g, resume it    
        if _g_.m.lockedg != nil {
            stoplockedm()
            execute(_g_.m.lockedg, false) // Never returns.
        }

        // ...

        var gp *g
        var inheritTime bool

        // ...

        // try getting from global queue
        if gp == nil {
            if _g_.m.p.ptr().schedtick%61 == 0 && sched.runqsize > 0 {
                lock(&sched.lock)
                gp = globrunqget(_g_.m.p.ptr(), 1)
                unlock(&sched.lock)
            }
        }

        // try getting from local queue
        if gp == nil {
            gp, inheritTime = runqget(_g_.m.p.ptr())
            if gp != nil && _g_.m.spinning {
                throw(``schedule: spinning with local work'')
            }
        }
    
        // nothing to run, call findrunnable
        if gp == nil {
            gp, inheritTime = findrunnable()
        }
        
        execute(gp, inheritTime)
    }

    func findrunnable() (gp *g, inheritTime bool) {
        _g_ := getg()

        // ...
        // local runq
        if gp, inheritTime := runqget(_g_.m.p.ptr()); gp != nil {
            return gp, inheritTime
        }

        // global runq
        if sched.runqsize != 0 {
            lock(&sched.lock)
            gp := globrunqget(_g_.m.p.ptr(), 0)
            unlock(&sched.lock)
            if gp != nil {
                return gp, false
            }
        }

        // Poll network.
        if netpollinited() && sched.lastpoll != 0 {
            if gp := netpoll(false); gp != nil {
                injectglist(gp.schedlink.ptr())
                casgstatus(gp, _Gwaiting, _Grunnable)
                return gp, false
            }
        }
        
        // ...
        
        // random steal from other P's
        for i := 0; i < int(4*gomaxprocs); i++ {
            _p_ := allp[fastrand1()%uint32(gomaxprocs)]
            var gp *g
            if _p_ == _g_.m.p.ptr() {
                gp, _ = runqget(_p_)
            } else {
                stealRunNextG := i > 2*int(gomaxprocs)
                gp = runqsteal(_g_.m.p.ptr(), _p_, stealRunNextG)
            }
            if gp != nil {
                return gp, false
            }
        }

        // ...
    }

    func newproc(siz int32, fn *funcval) {
        argp := add(unsafe.Pointer(&fn), sys.PtrSize)
        pc := getcallerpc(unsafe.Pointer(&siz))
        systemstack(func() {
            newproc1(fn, (*uint8)(argp), siz, 0, pc)
        })
    }

    func newproc1(fn *funcval, argp *uint8, narg int32,
                  nret int32, callerpc uintptr) *g {
        _g_ := getg()

        // ...

        _p_ := _g_.m.p.ptr()
        newg := gfget(_p_)
        if newg == nil {
            newg = malg(_StackMin)
            casgstatus(newg, _Gidle, _Gdead)
            allgadd(newg)
        }

        // configure newg
        // ...

        runqput(_p_, newg, true)

        // ...

        return newg
    }
\end{verbatim}

\section{Go channel}

Go中有两种chan:分别是单个的channel和有buffer的channel;有两种操
作: \verb+x <- c+,  \verb+c <- x+,分别是向其中添加内容和取出内容。构
建chan的操作在代码生成时被翻译为 \verb+runtime.makechan+函数,添加和取出
则分别翻译为 \verb+runtime.chansend+、 \verb+runtime.chanrecv+。这几个函
数在chan.go中实现。

Go语言的channel实现并非针对高性能并行设计,而是为了实现高度并发环境下的
通信设计的\cite{pike2012},它只是掩盖了底层的线程、锁等机制,提供一个易
于使用的通信模式。它类似Unix的管道,在进程间提供了最基础的通信服务,但
是比管道拥有更多的特性,比如支持类型。但是它不直接支持大部分并行模式,
如基于共享变量的并行模式及广播等模式(不过,由于Go是系统级语言,可以使
用Go实现这样的库或工具)。实现高性能并行主要需要靠用户的正确使用。

底层的执行基于进程,因此全部的通信基于内存复制而非内存共享。当然,由于
可以在channel中传递指针,亦即传递一块内存,事实上内存共享模式也是支持
的。我们后面再分析效率上的问题。

现在,我们具体看一下Go中chan的实现机制\cite{vyukov2014}。

Channel实现中信息由记录类型Hchan保存,其中记录如下一些信息:channel容
量、类型、元素buffer、接收队列、发送队列。

Go中的channel细分可以分为三种类型。下面是三种类型分别的构建方式:

\begin{verbatim}
    c1 := make(chan int)
    c2 := make(chan int, 10)
    c3 := make(chan struct{})
\end{verbatim}

具体分析三种channel:

注意到,第一种channel中只有一个空位,即表示内容传输始终是一对一的,只
要运行时确定一对发送者和接收者(各自取队首的g),然后直接转交,进行一
次内存复制即可。由于没有竞争,这里使用最简单的加锁实现,不会影响性能。
如果没有匹配,则将一个sudog加入发送者/接收者队列。sudog是可以被复用的,
它负责等待配对者出现而被唤醒。

第二种channel有一个元素buffer,这是个传统的生产者消费者模式,为了性能,
给整个buffer加锁是不理想的。于是这里采用了CAS操作来实现无锁访问。以
send为例,在buffer未满的情况下,各个sender竞争写sendx,竞争到的进行一
次(不需要加锁的)写操作。读也是一样。如果buffer满了,则休眠等待唤醒。

第三种和第二种是类似的,只是不需要存储数据。它就作为信号量来使用。

自然,我们还需要实现一个高效的select。首先对所有channel进行shuffle。之
后先对每一个channel尝试执行非阻塞的通信,如果失败(即等待的channel全部
阻塞),则在每个channel的等待列表中加入自己,并阻塞,等待之后被唤醒。
整个操作不需要对channel全部加锁,在很多情况下都不需要访问每个channel。

有关Go的内存模型,更为详细的内存访问时机的讨论,还可以阅读\cite{mem}。

知道了机制,我们可以再来看一下具体的代码。

用来保存channel的数据结构叫做\verb+hchan+:

\begin{verbatim}
    type hchan struct {
        qcount   uint           // total data in the queue
        dataqsiz uint           // size of the circular queue
        buf      unsafe.Pointer // points to an array of dataqsiz elements
        elemsize uint16
        closed   uint32
        elemtype *_type // element type
        sendx    uint   // send index
        recvx    uint   // receive index
        recvq    waitq  // list of recv waiters
        sendq    waitq  // list of send waiters
        lock     mutex
    }
\end{verbatim}

构建channel的函数\verb+makechan+即返回一个新的hchan对象。send和recv的
代码分别如下:

\begin{verbatim}
    func chansend(t *chantype, c *hchan, ep unsafe.Pointer,
                  block bool, callerpc uintptr) bool {
        // ...

        // check for failed operation without acquiring the lock
        if !block && c.closed == 0 &&
           ((c.dataqsiz == 0 && c.recvq.first == nil) ||
            (c.dataqsiz > 0 && c.qcount == c.dataqsiz)) {
            return false
        }

        // ...

        lock(&c.lock)

        // if there is a sudog waiting in recvq
        if sg := c.recvq.dequeue(); sg != nil {
            send(c, sg, ep, func() { unlock(&c.lock) })
            return true
        }

        // if channel buffer not full
        if c.qcount < c.dataqsiz {
            qp := chanbuf(c, c.sendx)
            typedmemmove(c.elemtype, qp, ep)
            c.sendx++
            if c.sendx == c.dataqsiz {
                c.sendx = 0
            }
            c.qcount++
            unlock(&c.lock)
            return true
        }

        // ...
        
        // buffer is full, block
        gp := getg()
        mysg := acquireSudog()
        mysg.releasetime = 0
        if t0 != 0 {
            mysg.releasetime = -1
        }
        mysg.elem = ep
        mysg.waitlink = nil
        mysg.g = gp
        mysg.selectdone = nil
        gp.waiting = mysg
        gp.param = nil
        c.sendq.enqueue(mysg)
        goparkunlock(&c.lock, "chan send", traceEvGoBlockSend, 3)

        // someone woke us up.
        gp.waiting = nil
        gp.param = nil
        if mysg.releasetime > 0 {
            blockevent(int64(mysg.releasetime)-t0, 2)
        }
        releaseSudog(mysg)
        return true
    }

    func chanrecv(t *chantype, c *hchan, ep unsafe.Pointer, 
                  block bool) (selected, received bool) {
        // fairly similar to chansend
    }
\end{verbatim}

\bibliography{ref}

\end{document}
